\chapter[Problem of Consistency]{The Problem of Consistency in Axiomatics as a Logical Decision Problem}

\section{Formal Axiomatics}

\subsection{Relationship of formal to contentual axiomatics; issue of
  consistency; arithmetization}

\paragraph{} The state of research in the foundations of mathematics
is characterized by three kinds of investigations:
\begin{enumerate}
\item the development of the axiomatic method (especially with
  applications to geometry);
\item the founding/grounding/re-working of analysis with today's
  rigorous methods;
\item investigations into the foundations of number theory and set theory.
\end{enumerate}

\paragraph{Problem of the infinite} Hilbert and Bernays explain
``because of new sharpened [more rigorous] methodological
requirements'' we will be forced to address ``the deeper problem of
the infinite in a new way''. But what, exactly, these ``new
methodological requirements'' are remains vague.

The ``problem of the infinite'' sounds strange to modern ears. David
Hilbert's ``On the infinite'' discusses this problem in the first four
paragraphs, namely that infinity could re-appear in unexpected ways.

Hilbert and Bernays propose to address this ``problem'' through their
investigation [treatment?] of axiomatics.

\paragraph{The term ``Axiomatics''}
The term ``axiomatics'' has a broad sense, and a narrow sense.
In the broadest sense: the notions and presuppositions are stated
first, then the theory is developed logically. Examples of this
``broader'' style of axiomatics include: Euclid's \textit{Elements},
Newton's mechanics, and Clausius's thermodynamics.

\paragraph{} Hilbert's \textit{Foundations of Geometry} (1899)
develops the notion of axiomatics further. From the
``factual and conceptual'' [representational?] subject matter which
gives us the basic notions of the theory, we retain only the essence
of the theory formulated as axioms, and ignore [abstract away] all
other content.

There is a further quality which gives us the narrowest sense of
axiomatics, which Hilbert and Bernays call \define{Existential Form}
of axiomatics.
For them, this distinguishes the ``axiomatic method'' from the
``constructive or genetic method'' for grounding a theory.

\subparagraph{Genetic method}
The constructive/genetic method ``introduces the objects of a theory
as a genus of things''.

The notion of a ``genus'', Brouwer conventionally calls
``species''. (This is an issue of translation: the German word for
Brouwer's ``species'' is \textit{Gattung}, which translates to English
as ``genus''. Translation seldom forms a commutative diagram.)

Hilbert and Bernays contrast this with the axiomatic method, which
works with a \emph{fixed} ``domain of discourse'' given at the outset.
This is what predicates ``work on''.

\subparagraph{About ``existential''}\label{subpar:ch1:existential-form:about-existential}
We should note that the notion of ``existential'' (in
``existential axiomatics''), Bernays writes in a later article (in
1950) that ``Thus, in mathematics, we have no reason to to assume any
meaning of `existence' that would be fundamentally different from
that of `the validity of axiomatic relations'.''

Consequently, we should bear in mind the discussion of ``existence''
and ``existential'' (as adjectives for a theory or formalism) carries
this connotation of ``the validity of axiomatic relations''.

\paragraph{}
Hilbert and Bernays stress that the assumption that there exists a
``domain of individuals'' is an additional axiom (implicit, seldom
explicitly stated) for formal axiomatics, except in the ``trivial''
situation where the domain of individuals is finite. That is, when
there are only finitely many individuals in the domain.

\paragraph{Formal Axiomatics}
This ``sharpened'' [more sophisticated? more refined? more rigorous]
form of axiomatics, which ignores its subject matter and requires the
existential form, Hilbert and Bernays refer to as merely
\define{Formal Axiomatics}.

The characteristic requirement of formal axiomatics: we need a
\emph{proof of consistency}. Recall (\pilcrow\ref{subpar:ch1:existential-form:about-existential})
a proof of consistency amounts to proving the existence of a domain of
discourse.

On the other hand, the \emph{contentual axiomatics} introduces the basic
notions by referring to common experience and presents its ``first principles''
either as evident facts (which you can verify for yourself) or
formulates them as extracts from ``experience-complexes''.

So contentual axiomatics supposes we have stumbled across the correct
laws of nature, and will reinforce this belief with the success of the
theory. (This seems like the approach most natural sciences take.)

\paragraph{}
Just as contentual axiomatics requires certain kinds of evidence,
formal axiomatics too requires certain kinds of evidence for (a)
pursuing deductions, and (b) the proof of consistency.

The fundamental difference is that contentual axiomatics allow
evidence based on a certain special kind of epistemic status, whereas
formal axiomatics has abstracted this all away and therefore requires
``the same kind'' of evidence.

\paragraph{}
What is the relation between contentual and formal axiomatics?

\paragraph{}
Formal axiomatics requires contentual axiomatics as a necessary
adjunct (``adjunct'' being my choice of words). How else would we
determine the intended interpretation?

Hilbert and Bernays phrase it as: contentual axiomatics gives us
guidance for determining the ``right formalism'', and some
instructions for how to apply a theory to a ``domain of actuality''.

\paragraph{}
We cannot rest content with contentual axiomatics, because in science
we are not reproducing every aspect of the actual state-of-affairs in
its entirety, but instead we work with a simplifying
idealization. (This works for science because we successively refine
our understanding.)

Rather, a theory formulated using contentual axiomatics does not get a
grounding from the self-evident truth of its axioms, nor from
experience. Instead, it receives a grounding only by showing its
idealization is free from contradiction.

\paragraph{}
We're therefore forced to investigate the consistency of such
theoretical systems without considering ``actuality'' [empirical evidence].
And we thus arrive at the standpoint of formal axiomatics.

\paragraph{Method of Arithmetization}
Hilbert and Bernays introduce the ``method of arithmetization'':
\begin{itemize}
\item the objects of the theory is translated to numbers or number systems;
\item the axioms of the theory is translated to equations or inequalities;
\item the result of translating the axioms results in equations or
  inequalities which are arithmetical identities or provable results.
\end{itemize}
This method boils down to proving consistency of a theory assuming
that arithmetic is consistent [``the validity of arithmetic''].
``And so we must ask what kind of validity this amounts to,'' Hilbert and Bernays ask.

(I suspect this is also why Hilbert's metatheory is a finitary,
intuitive, contentual number theory: it is impossible to believe it
inconsistent.)

\paragraph{}
Can't we \emph{directly} prove the consistency of a theory?

Hilbert and Bernays will attempt to answer this question, while using
it as a pretext for introducing ``logical symbolism'' \emph{and}
simultaneously getting a better idea for the general form of the
problem of consistency.

\subsection{Geometrical axioms as an example}

\paragraph{Example: Planar geometry}
We start with the example of \emph{plane geometry}.
The domain of individuals will consist entirely of points.

There are two primitive relations.
\begin{enumerate}
\item Instead of the relation ``$x$ and $y$ lie on a straight line $g$'', we
will take the relation
\begin{equation*}
  Gr(x,y,z)\sim%
\begin{pmatrix}
  \mbox{$x$, $y$, $z$ lie in}\\
  \mbox{a straight line}
\end{pmatrix}
\end{equation*}
\item The second primitive relation encodes our intuition of
``between-ness'':
\begin{equation*}
Zw(x,y,z)\sim%
\begin{pmatrix}
  \mbox{$x$ lies in}\\
  \mbox{between $y$ and $z$}
\end{pmatrix}
\end{equation*}
\end{enumerate}
We also will take equality $=$ for a primitive notion of identity,
writing $x=y$ for ``$x$ is identical to $y$''.

\paragraph{Quantifier notation}
Now, we can introduce the notation for quantifiers. Hilbert and
Bernays write this as, for a unary predicate $P[x]$,
\begin{equation*}
  \begin{array}{rl}
\forall{x}{P[x]}&\mbox{means ``All $x$ have the property $P[x]$''}\\
\exists{x}{P[x]}&\mbox{means ``Some $x$ has the property $P[x]$''}
  \end{array}
\end{equation*}
The $\forall{x}$ is the universal quantifier, the $\exists{x}$ is the
existential quantifier, and the variable $x$ appearing in either one
is considered a \emph{bound variable} (Hilbert and Bernays's term).

\paragraph{Logical connectives}
Hilbert and Bernays introduce the usual logical connectives:
\begin{enumerate}
\item $A\land B$ means ``$A$ and $B$ both hold''
\item $A\lor B$ means ``either $A$ or $B$ (or both) hold'' (Hilbert
  and Bernays write: ``or in the sense of \textit{vel\/}'')
\item $\Not{A}$ means ``not $A$''; Hilbert and Bernays write ``$x\neq y$''
for ``$\Not{x=y}$''.
\end{enumerate}

\paragraph{Implication}
The connective ``$\implies$'' is used to indicate logical
implication. Specifically ``if the first proposition holds, then so
does the second''.

Specifically $\meta{A}\implies\meta{B}$ is false if and only if
$\meta{A}$ is true and $\meta{B}$ is false. Otherwise,
$\meta{A}\implies\meta{B}$ is true. (NOTE: Hilbert and Bernays appear
to use fraktur letters $\meta{A}$, $\meta{B}$ for metavariables.)

\paragraph{}
The universal quantifier and implication combines in the intended
way. The example given is
\begin{equation*}
\Forall{x}\Forall{y}{(\meta{A}(x,y)\implies\meta{B}(x,y))}.
\end{equation*}
This means ``For any $x$ and $y$, if $\meta{A}(x,y)$ holds, then so
too does $\meta{B}(x,y)$ hold''.

\paragraph{}
Hilbert and Bernays give precedence to the connectives, which differs
than modern conventions: $\implies$ has lower precedence than $\land$
(as is the case in modern conventions), but $\land$ has lower
precedence than $\lor$ (contrary to modern conventions).

The quantifiers have higher precedence than connectives. But
universal quantification has lower precedence than existential
quantification.

Wherever there may be ambiguity, parentheses will be inserted.

\paragraph{}
Now we can begin presenting the axioms of planar geometry.

\paragraph{}\label{paragraph:ch1:planar-geometry-axiom-start}
Note that this presentation is not identical to the one found in
Hilbert's \textit{Foundations of Geometry} (1899).

\begin{center}
  I. Axioms of Connection
\end{center}

\begin{enumerate}
\item $\Forall{x}\Forall{y}{Gr(x,x,y)}$ (``for all $x$ and $y$ we have
  $x$, $x$, $y$ lie on a straight line'')
\item $\Forall{x}\Forall{y}\Forall{z}{(Gr(x,y,z)\implies Gr(y,x,z)\land Gr(x,z,y))}$
(``whenever $x$, $y$, $z$ lie on a straight line, then so too do $y$,
  $x$, $z$ and $x$, $z$, $y$'')
\item $\Forall{x}\Forall{y}\Forall{z}\Forall{u}{(Gr(x,y,z)\land
  Gr(x,y,z)\land u\neq z\implies Gr(x,z,u))}$
\item\label{axiom:planar-geometry:4} $\Exists{x}\Exists{y}\Exists{z}\Not{Gr(x,y,z)}$ (There exists
  $x$, $y$, $z$ which do not lie on the same line)
\end{enumerate}

\paragraph{}
These axioms loosely correspond to Axioms I found in Hilbert's book on
geometry.

\begin{center}
  II. Axioms of Order
\end{center}

\begin{enumerate}[resume]
\item $\Forall{x}\Forall{y}\Forall{z}{(Zw(x,y,z)\implies Gr(x,y,z))}$
\item $\Forall{x}\Forall{y}\Not{Zw(x,y,y)}$
\item $\Forall{x}\Forall{y}\Forall{z}{(Zw(x,y,z)\implies Zw(x,z,y)\land\Not{Zw(y,x,z)})}$
\item\label{axiom:planar-geometry:II:4} $\Forall{x}\Forall{y}{(x\neq y\implies\Exists{z}{Zw(x,y,z)})}$
  (for any distinct points $x$ and $y$, there exists a third point $z$
  such that $x$ lies between $y$ and $z$)
\item\label{axiom:planar-geometry:II:5} $\Forall{x}\Forall{y}\Forall{z}\Forall{u}\Forall{v}{\bigl(\Not{Gr(x,y,z)}\land Zw(u,x,y)\land\Not{Gr(v,x,y)}\land\Not{Gr(z,u,v)}\implies\Exists{w}{(Gr(u,v,w)\land Zw(u,x,y)\lor\Not{Gr(v,x,y)})}\bigr)}$
\end{enumerate}

\paragraph{}
These axioms in group II roughly correspond to Hilbert (1899)'s Axioms II.

\begin{center}
  III. Axioms of Parallels
\end{center}

\paragraph{}
We omit the axioms of congruences, since the axiom of parallels is
stronger. Informally stated, ``For any line passing through two
points, for any point not on the line, there exists a line passing
through the third point which does not intersect the original line.''

\paragraph{Notation}
We introduce the notational abbreviation
\begin{equation*}
par(x,y;u,v) := \Not{\Exists{w}{(Gr(x,y,w)\land Gr(u,v,w))}}
\end{equation*}
(the ``$:=$'' symbol is mine, not Hilbert and Bernays's).

\paragraph{}
This informally means ``There is no point $w$ which lies on the line
passing through $x$ and $y$, and also lies on the line passing through
$u$ and $v$.''

\paragraph{Axiom}
The axiom of parallels now reads:
\begin{equation*}
\Forall{x}\Forall{y}\Forall{z}\left(\Not{Gr(x,y,z)}\implies\Exists{u}{\Bigl(par(x,y;z,u)\land\Forall{v}{\bigl(par(x,y;u,v)\implies Gr(x,u,v)\bigr)}\Bigr)}\right)
\end{equation*}


\paragraph{}
If we consider all the axioms presented here, we could conjoin them
altogether into one giant axiom which Hilbert and Bernays write as
\begin{equation*}
\meta{A}(Gr,Zw).
\end{equation*}

\paragraph{}
Similarly, we can write any theorem about planar geometry as a formula
involving only $Gr$ and $Zw$ as
\begin{equation*}
\meta{S}(Gr,Zw)
\end{equation*}
(That's a fraktur ``S'', by the way, since the German word for
proposition is \textit{Satz\/}.)

\paragraph{}
Yet everything we have presented has been presented using
\emph{contentual axiomatics}. And in contentual axiomatics, every
basic relation is viewed we experience, or intuitively understand
contentually, and thus ``as something contentually determines'' about
which the sentences of the theory contain assertions.

\subsection{Abstract logical formulation of the axiomatics}

\paragraph{}
In formal axiomatics, the basic\footnote{``Basic'' as in ``primitive notion''.} relations
are not contentually determined from the start.
Instead, they ``receive their determination only \emph{implicitly} from
the axioms.''

It seems that Hilbert and Bernays have a nascent notion of proof-proof
theoretic semantics.

\paragraph{}
In the example of planar geometry, when we used phrases like ``lies between''
or ``lies on a line containing'', these are just informal heuristics
to aid the reader.

In reality, the primitive relations play the role of distinct
predicate variables.

\paragraph{Terminology}
Hilbert and Bernays use the word ``predicate'' in a broader sense to
refer to unary predicates, binary predicates, \dots, $n$-ary predicates.

\paragraph{}
So far, in axiomatic geometry, we have introduced two distinct
ternary predicate variables $R(x,y,z)$ and $S(x,y,z)$.

\paragraph{}
The axiom system consists of a ``requirement'' (stipulation?) on two
such predicates as satisfying the formula $\meta{A}(R,S)$.

\subparagraph{Equality}
Further, the relation ``$x=y$'' cannot be determined axiomatically but
contentually. This does not contradict anything from the methodological
perspective Hilbert and Bernays advocate.

The contentual determination of identity does not depend on ``any
particular conception in'' [or, ``the conceptual content of''] the
field of axiomatic investigation. Rather, it concerns distinguishing
individuals [in the domain of discourse].

The idea that there's something \emph{semantic} to equality/identity
is not surprising. Tarski discovered the same thing.

\paragraph{}
From this perspective, a sentence $\meta{S}(Gr,Zw)$ has the following
interpretation: for \emph{any} predicates $R(x,y,z)$ and $S(x,y,z)$
satisfying $\meta{A}(R,S)$, we necessarily have $\meta{S}(R,S)$ hold
as well. That is to say,
\begin{equation*}
\meta{A}(R,S)\implies\meta{S}(R,S)
\end{equation*}
is a true proposition. This transforms a theorem from planar geometry
into a sentence of predicate logic.

\paragraph{Consistency}
The problem of consistency can be phrased as such: are there
predicates $R$ and $S$ satisfying $\meta{A}(R,S)$? Or are there no
such predicates? That is, $\meta{A}(R,S)\implies\Not{\meta{A}(R,S)}$
for every binary predicates $R$ and $S$?

Consistency of planar geometry becomes a question of pure logic.

\section{The Entscheidungsproblem}

\subsection{Validity and satisfiability}

\paragraph{Problem statement}
Is there a decidable procedure for determining if a formula is valid
or satisfiable?

This is the Entscheidungsproblem, which Church answered in the negative.

\paragraph{}
The formulas Hilbert and Bernays work with have no free variables, and
are built `in the obvious way'.

\paragraph{Definitions}
\begin{enumerate}
\item A formula is called \define{Valid} if it is a true proposition
  for every ``determination of predicates''.
\item A formula is called \define{Satisfiable} if it is a true
  proposition for \emph{some} ``determination of predicates''.
\end{enumerate}

\paragraph{Examples}
Some valid formulas include:
\begin{itemize}
\item $\Forall{x}F(x)\;\land\;\Forall{x}G(x)\quad\implies\quad\Forall{x}{(F(x)\land G(x))}$
\item $\Forall{x}P(x,x)\quad\implies\quad\Forall{x}\Exists{y}{P(x,y)}$
\item $\Forall{x}\Forall{y}\Forall{z}{\bigl(Q(x,y)\land y=z\implies Q(x,z)\bigr)}$
\end{itemize}

\paragraph{Examples}
Some satisfiable formulas include:
\begin{itemize}
\item $\Exists{x}(F(x))\;\land\;\Exists{x}{\Not{(F(x))}}$
\item $\Forall{x}\Forall{y}{\bigl(P(x,y)\land P(y,x)\;\implies\; x=y\bigr)}$
\item $\Forall{x}\Exists{y}{Q(x,y)}\quad\land\quad\Exists{y}{\Forall{x}\Not{Q(x,y)}}$
\end{itemize}

\paragraph{}
To see this, consider the following examples:
\begin{itemize}
\item $F(x)$ means ``$x$ is even''
\item $P(x,y)$ means ``$x\leq y$''
\item $Q(x,y)$ means ``$x\leq y$ and $y\neq 1$''
\end{itemize}

\paragraph{}
Simultaneous with a ``determination of predicates'', we must fix a
``domain of individuals''.

\subparagraph{``Hidden variable''?} Hilbert and Bernays explain that
the domain of individuals plays a role analogous to a ``hidden variable''.
The exact meaning of this eludes me at the moment.

\subparagraph{Invariance of domain of individuals}
If we have a bijection from one domain of individuals to another, then
nothing substantially changes. Therefore, only the \emph{number of individuals}
is the important property of the domain of individuals.

\paragraph{Questions}
There are three questions we can meaningfully ask:
\begin{enumerate}
\item The validity of a formula for all domains of individuals (resp.,
  the satisfiability for some domain of individuals)?
\item The question of validity (resp., satisfiability) for a given
  domain of individuals?
\item For which domains of individuals is a formula valid (resp., satisfiable)?
\end{enumerate}

\subsection{Decision for finite domains of individuals}

\paragraph{Empty domain excluded}
Hilbert and Bernays exclude domains of individuals containing zero
individuals. 

\paragraph{Graphs of predicates}
We also need to consider ``graphs'' of predicates, i.e., enumerating
the truth-value for predicates for every possible combination of
individuals.

\paragraph{}
For finite domains, this boils down predicate logic to a purely
combinatorial state of affairs.

\paragraph{}
If we have a domain with $n$ individuals, then the graph for a $k$-ary
predicate requires $n^{k}$ rows. This also means that there are
$2^{n^{k}}$ possible distinct $k$-ary predicates (since each row could
be true or false).

\paragraph{}
If we have a formula with predicates $R_{1}$, \dots, $R_{t}$ each with
arity $k_{1}$, \dots, $k_{t}$ (respectively), then we see there are
$2^{(n^{k_{1}}+\dots+n^{k_{t}})}$ distinct possible graphs for such a formula.

If our axiomatic system has basic predicates $R_{1}$, \dots, $R_{t}$ each with
arity $k_{1}$, \dots, $k_{t}$ (respectively), then there are
$2^{(n^{k_{1}}+\dots+n^{k_{t}})}$ possible realizations of the system.
Hilbert and Bernays call each possibility a \define{Predicate System}.

\paragraph{}
Validity for a formula means that for all $2^{(n^{k_{1}}+\dots+n^{k_{t}})}$
predicate systems, the formula is always true.

Satisfiability for a formula means that there exists one out of the
$2^{(n^{k_{1}}+\dots+n^{k_{t}})}$ possible predicate systems such that
the formula is true.

\paragraph{Examples}
Consider the formulas:
\begin{enumerate}
\item $\Forall{x}{P(x,x)}\quad\implies\quad\Forall{x}\Exists{y}{P(x,y)}$
\item $\Forall{x}\Forall{y}{(P(x,y)\land P(y,x)\quad\implies\quad x=y)}$
\end{enumerate}
We will check the first formula is valid, the second formula is satisfiable.

\paragraph{}
We work with a domain with two individuals, which we denote by $1$ and $2$.
Here our axiomatic system has $t=1$ predicate $P(-,-)$ which has
binary (i.e., $k_{1}=2$). Since there are two individuals, $n=2$, and
that means that we have $2^{(n^{k_{1}})}=2^{(2^{2})}=16$ possible
predicate systems.

\paragraph{}
Observe then that $\Forall{x}{P(x,x)}$ abbreviates
\begin{equation*}
P(1,1)\land P(2,2).
\end{equation*}
Similarly, $\Forall{x}\Exists{y}{P(x,y)}$ abbreviates
\begin{equation*}
(P(1,1)\lor P(1,2))\land(P(2,1)\lor P(2,2)).
\end{equation*}
Then the first formula we're considering corresponds to:
\begin{equation*}
\bigl(P(1,1)\land P(2,2)\bigr)\quad\implies\quad\bigl((P(1,1)\lor P(1,2))\land(P(2,1)\lor P(2,2))\bigr).
\end{equation*}

\paragraph{}
The implication is true for those ``predicate systems'' where
\begin{equation*}
P(1,1)\land P(2,2)
\end{equation*}
is true. After all, this is just $A\land B\land[(A\land B)\implies((A\lor C)\land(B\lor D))]$
(write down its truth table, square brackets used for clarity).

Similarly, when
\begin{equation*}
P(1,1)\land P(2,2)
\end{equation*}
is false but
\begin{equation*}
(P(1,1)\lor P(1,2))\land(P(2,1)\lor P(2,2))
\end{equation*}
is true, we see the formula is true. To see this, write down the truth
table for $P(1,1)$, $P(1,2)$, $P(2,1)$, $P(2,2)$ and select the
rows for which $P(1,1)\land P(2,2)$ is false but
\begin{equation*}
(P(1,1)\lor P(1,2))\land(P(2,1)\lor P(2,2))
\end{equation*}
is true.

\paragraph{Second formula}
The second of the two examples we're considering,
$\Forall{x}\Forall{y}{(P(x,y)\land P(y,x)\quad\implies\quad x=y)}$,
over the same domain with two individuals can be expanded to the
formula:
\begin{equation*}
\begin{array}{rcl}
(P(1,1)\land P(1,1)\implies 1=1)&\land&(P(1,2)\land P(2,1)\implies 1=2)\quad\land\\
(P(2,1)\land P(1,2)\implies 2=1)&\land&(P(2,2)\land P(2,2)\implies 2=2).
\end{array}
\end{equation*}

\paragraph{}
The first and last conjunctand is always true (since $1=1$ and $2=2$
always is true). So this is really boiling down to
\begin{equation*}
(P(1,2)\land P(2,1)\implies 1=2)\land(P(2,1)\land P(1,2)\implies 2=1).
\end{equation*}
But since $2\neq1$ (and $1\neq 2$), we need $P(1,2)\land P(2,1)$ to be false.

\paragraph{}
Therefore, the situations when the formula is satisfiable boils down
to examining when $P(1,2)\land P(2,1)$ is false. When $P(1,2)\land P(2,1)$
evaluates to ``true'', the entire formula is false. But there are 4
situations when $P(1,2)\land P(2,1)$ is false, and those 4 situations
describe when the formula is satisfiable.

\paragraph{}
These examples demonstrate the combinatorial character of the
Entscheidungsproblem on a finite domain of individuals.

\subparagraph{}
Observe for finite domains of individuals that a formula $\meta{F}$ is
valid if and only if $\Not{\meta{F}}$ is unsatisfiable.

\subparagraph{}
Similarly, for a finite domain of individuals, a formula $\meta{F}$ is
satisfiable if and only if $\Not{\meta{F}}$ is not valid. If
$\Not{\meta{F}}$ is valid, then there is no predicate system for which
$\Not{\Not{\meta{F}}}$ is true (and therefore $\meta{F}$ cannot be satisfiable).

\subsection{Method of exhibition}

\paragraph{}
We return to our problem of consistency, and we assume that we have
formed a single formula $\meta{A}$ for our axioms conjoined together.

\paragraph{}
For a given domain of individuals, the questions of satisfiability and
validity boil down to testing it out (at least, in principle).

If we have determined a formula is valid on the given domain of
individuals, then we may treat it as a ``model'' (their term) for the
theory where the axioms are satisfied. This also proves the
consistency of the theory. Hilbert and Bernays calls this the
``method of exhibition''.

\paragraph{}
As an example, let us consider the axioms for planar geometry (from
\pilcrow\ref{paragraph:ch1:planar-geometry-axiom-start} \textit{et seq.}).
We need to replace Axiom~\ref{axiom:planar-geometry:4} with
something weaker.
\begin{itemize}
\item[(4')] $\Exists{x}\Exists{y}(x\neq y)$ (there are at least two
  distinct points)
\end{itemize}

\paragraph{}
Hilbert and Bernays drop the Axiom of
order~\ref{axiom:planar-geometry:II:5} and replace it with something
else. First, we modify Axiom~\ref{axiom:planar-geometry:II:4} to
become
\begin{itemize}
\item[(8')] $\Forall{x}\Forall{y}{(x\neq y\quad\implies\quad \Exists{z}{(Zw(z,x,y))}\;\land\;\Exists{z}{(Zw(x,y,z))})}$
\end{itemize}
Then Hilbert and Bernays add the axiom
\begin{itemize}
\item[(9')] $\Forall{x}\Forall{y}\Forall{z}{(x\neq y\neq z\quad\implies\quad Zw(x,y,z)\land Zw(y,z,x)\land Zw(z,x,y))}$
\end{itemize}
However, Claus-Peter Wirth point out this is a flawed formalization of
the intended axiom
\begin{itemize}
\item[(9'')] $\Forall{x}\Forall{y}\Forall{z}{(x\neq y\neq z\land Gr(x,y,z)\quad\implies\quad Zw(x,y,z)\land Zw(y,z,x)\land Zw(z,x,y))}$
\end{itemize}

\paragraph{}
The axiom of parallels remains unchanged. When we conjoin them
altogether, we get a formula $\meta{A}'(R,S)$ whereas we formerly had
$\meta{A}(R,S)$.

Veblen demonstrates that there exists a finite model consisting of 5
distinct individuals for $\meta{A}'(R,S)$. This is done as follows:
$Gr(x,y,z)$ is always true. And $Zw$ is determined according to the
three rule:
\begin{enumerate}
\item For any triple $x$, $y$, $z$ where at least two of the
  individuals coincide, we have $Zw(x,y,z)$ be false. This means there
  are $\binom{5}{3}\cdot 3! = 10\cdot 6 = 60$ possible triples where $Zw(x,y,z)$
  could be true
\item Among the possible permutations of these 60, there are 6
  orderings to each of the $\binom{5}{3}$ ways of choosing 3 distinct
  individuals among the 5 of the domain. Among these 6 orderings, we
  insist that 2 will give $Zw$ being true, and the other 4 give $Zw$
  being false.
\item For any distinct pair of individuals, they appear as the first
  two entries in a triplet where $Zw$ is true \emph{and} as the last
  two entries in a triplet where $Zw$ is true.
\end{enumerate}

\paragraph{}
Hilbert and Bernays work out a bit of the combinatorics of the cases
in this paragraph. But the first condition (if there's a duplicate
among $x$, $y$, $z$, then $Zw(x,y,z)$ is false) is ``just stipulated''.

They deduce that there are 20 triples among the possible 60 where $Zw$
is true. Furthermore, they are the triplets
\begin{enumerate}
\item $(1,2,5)$
\item $(1,5,2)$
\item $(1,3,4)$
\item $(1,4,3)$
\end{enumerate}
and all those obtained from them by applying the cyclic permutation $(1\;2\;3\;4\;5)$.

\paragraph{}
``It's easy to see this is consistent'', we're told. And it's all done
by the method of exhibition.

\paragraph{}
This is the method of exhibition, and it's really useful for
independence proofs.

If we want to prove $\meta{S}$ is independent of the axiom system
$\meta{A}$, then we just prove $\Not{\meta{S}}\land\meta{A}$ is consistent.

\section{The issue of consistency for infinite domains of individuals}

\subsection{Formulas that are not satisfiable in the infinite; the
  number series as a model}

\paragraph{}
So, we've seen the method of exhibition and its usefulness.

But for infinite domains of individuals, can we still use the method
of exhibition? It's not clear immediately (one way or the other; i.e.,
that it's applicable or not).

However, it is clear that the ``predicate systems'' (as extended truth
tables) runs into difficulties. Hilbert and Bernays tell us the
collection of possible predicate systems is no longer a
``comprehensible-manifold'' [Ger.: \textit{Mannigfaltigkeit\/}, a
Kantian term].  %                          Mannigfaltigkeit

\subparagraph{Manifold}
It also seems to be ``multiplicity''. In Kant's \textit{Critique of Pure Reason},
it appears in 36 places; e.g.,
\begin{quote}
Further, this science cannot be terribly extensive, for it does not deal
with objects of reason, whose multiplicity [\textit{Mannigfaltigkeit\/}]
is infinite, \dots (B23)
\end{quote}
By my count, it appears 45 times in Kant's \textit{Prolegomena}.

I am not sure what to make of this.

\subparagraph{Example}
Not one to be deterred or discouraged, Hilbert and Bernays offer the
following example: consider the axiomatic system consisting of a
binary predicate $R$ and the following axioms:
\begin{enumerate}
\item $\Forall{x}\Not{R(x,x)}$
\item $\Forall{x}\Forall{y}\Forall{z}{(R(x,y)\land R(y,z)\implies R(x,z)}$
\item $\Forall{x}\Exists{y}{R(x,y)}$
\end{enumerate}

\paragraph{}
Axioms (1) and (3) combine to tell us that for each $x$, there exists
a $y\neq x$ such that $R(x,y)$.

So if we know there is at least one object $a$ in our domain of
individuals, we can immediately infer there exists a $b\neq a$ such
that $R(a,b)$. But this observation cascades iteratively: there is a
$c\neq a$, $c\neq b$ such that $R(b,c)$, and so on and so forth. In
fact, there are infinitely many individuals in the domain of discourse
because this process does not terminate.

We can prove consistency by taking the integers as the domain of
individuals, and interpret $R(x,y)$ as ``$x$ is strictly less than $y$''.


\paragraph{Second Example}
Consider the axioms
\begin{enumerate}
\item $\Exists{x}\Forall{y}\Not{S(y,x)}$
\item $\Forall{x}\Forall{y}\Forall{u}\Forall{v}{(S(x,u)\land S(y,u)\land S(v,x)\implies S(v,y)}$
\item $\Forall{x}\Exists{y}{S(x,y)}$
\end{enumerate}

\paragraph{}
We can show that this is not satisfiable on a finite domain of
individuals by combining axioms (1) and (3) together again.

If we interpret $S(x,y)$ as ``$y$ is the immediate successor of $x$'',
then we have a model of this system: the natural numbers\footnote{That
is to say, the non-negative integers.}.

\paragraph{Observation}
In these two examples, we see that the method of exhibition does not
prove the consistency of the axiom systems. Instead, it ``merely''
\emph{reduces consistency down to the consistency of number theory}.

\subparagraph{Critical property}
In the previous example, the method of exhibition required one
critical property: the required properties of numbers were concretely
verifiable. 

\subparagraph{Other domains}
Hilbert and Bernays note: for finite domains, we relied on integers
only for simplicity. But we could have reduced it to a different
finite domain of individuals, like letters.

\paragraph{}
But we run into the following difficulty: a concrete conception
[representation] for the integers is not possible (in the sense that
we can't write them all down).

Essentially we need to presuppose that the integers constitute a
domain of individuals, and therefore a ``complete totality''
[\textit{fertige Gesamtheit}].

\subsection{Problems of the infinite}

\paragraph{}
Frege demanded a proof of consistency before using a ``number series''
as a completed totality. For Frege, such a proof should be an
\emph{existence proof} (in the sense of a method of exhibition).

\subparagraph{Problem with Frege's approach}
The discussion here in Hilbert and Bernays is rather philosophical for
my tastes.

As I understand it (and I don't), Frege wanted to work in the domain
of ``all conceivable unary predicates'', and then form the ``number series''
out of them\dots somehow.

Hilbert and Bernays cite Russell and Zermelo for dispelling this idea.

Claus-Peter Wirth notes that Quine~\cite{quine1981mathematical}
re-examines Frege's strategy for defining numbers. The problems Frege
encounters is not from his generic strategy \emph{per se}, but from
the ambient collection of all conceivable predicates leads to problems
like Russell's paradox.

The upshot is: the use of [infinitely many] numbers cannot be used
willy-nilly. Problems arise from infinities.

\subparagraph{``Number series''}
The phrase ``number series'' appears to be the translation of
\textit{Zahlenreihe}, which I think has more of a meaning of a
``sequence of numbers''. This makes more sense to think of it as a
recursive generation of numbers (i.e., a procedure for enumerating
numbers), rather than a ``finished set'' of numbers.

Mueller~\cite{mueller2006grundlagen} translates the phrase as
``sequence of integers'', which supports this interpretation.

\paragraph{}
In light of these problems, we could consider using an infinite domain
of individuals which is not a mere product of thought.

We could try taking a domain from ``sense perception'' or ``physical reality''
instead. This runs into the problem that, upon closer inspection, it
eventually reduces to a \emph{perception} of an infinite domain. And
this perception boils down to modeling it using an infinite sequence
of numbers. So we're back where we started.

\paragraph{Zeno's paradox}
A typical example of this problem with infinities is Zeno's paradox.

Recall the paradox: a finite distance can be traversed in finite
time. The traversal includes infinitely many successive
subprocesses. The traversal of the first half, then the next quarter,
then the next eighth, and so on.

If we are considering an actual motion, then these subtraversals must
be genuine processes succeeding one another.

\paragraph{Essence of the paradox}
We usually refute this paradox with the argument that the sum of
infinitely many temporal intervals may converge producing a finite
duration.

But this misses the essential point of the paradox: the paradox lies
in the fact that an infinite succession, the completion of which we
could not accomplish in the imagination (either in practice or even in
principle), but should be accomplished in fact.

\paragraph{Using modern physics}
There are radical solutions to this paradox, Hilbert and Bernays note.
Modern physics suggests it is not physically meaningful to divide a
time interval into arbitrarily small subintervals.

Even continuum mechanics models a fluid using a continuum of
point-particles. 

\paragraph{}
One problem with this approach is that the mathematical model is based
on a \emph{simplifying idealization} of physical reality.

Another problem: the extrapolation from the mathematical model of
physical reality must be self-consistent.

Essentially, we end up back where we started.

\paragraph{}
In every case where we think we've found infinity directly exhibited
by perception or intuition (like musical notes on octaves, or the
color spectrum), we end up in a situation similar to the radical
solution to Zeno's paradox. Upon closer inspection, infinity is not
present at all: it emerges from mental extrapolation or interpolation.

\subsection{Proof of consistency as impossibility proof; method of arithmetization}

\paragraph{}
Upon these considerations, the existence of an infinity ``manifold''
[\textit{Gesamtheit\/}] cannot be determined by appealing to
non-mathematical objects. Instead, the question must be solved within
mathematics itself. But how should we begin such a solution?

We're asked to do the impossible: producing infinitely many
individuals is impossible in principle. Therefore an infinite domain
of individuals can only be indicated by its structure (i.e., through
relations holding among its elements).

\subparagraph{Vicious circle}
In other words, a proof must be given that for this domain certain
formal relations can be satisfied.

The existence of an infinite domain of individuals \emph{cannot be
represented in any way other than through the satisfiability of
certain logical formulas.}

But these are exactly the kind of formulas we were led to investigate
about the existence of an infinite domain of individuals.
And their satisfiability was demonstrated by the exhibition of an
infinite domain of individuals.

Thus we have been led into a vicious circle.

\paragraph{}
The method of exhibition was a means to prove the
consistency of an axiom system. We arrived here after recognizing that
(for a finite domain of individuals) the consistency of a formula is
equivalent to its satisfiability. 

\paragraph{}
The situation of an infinite domain of individuals is more complicated.
We still have the property that the axiom system $\meta{A}$ is
inconsistent if and only if it can prove $\Not{\meta{A}}$.

The difficulty? We no longer have a ``comprehensible'' supply of
combinations of graphs for the predicate variables. Consequently, from
$\Not{\meta{A}}$ being invalid, we can no longer conclude we have some
model of $\meta{A}$ at our disposal.

\paragraph{}
That is to say: the satisfiability of an axiom system is a sufficient
condition for proving its consistency, it is no longer
\emph{necessary} condition when using an infinite domain of individuals.

Ultimately, this leads Hilbert and Bernays to conclude that to prove
the consistency of an axiom system $\meta{A}$, we need to prove that
$\meta{A}$ cannot lead to a contradiction.

\subparagraph{Remark}
Much of the reasoning here stems from intuitionistic constraints. I
think \emph{classically}, Hilbert and Bernays is in error. But they
are trying to address intuitionistic skeptics.

\paragraph{}
From this perspective, the formalization of logical inference (by
Frege, Schr\"{o}der, Peano, and Russell) presents itself as an
appropriate means to our desired ends. 

\paragraph{Tasks before us}
We arrive at the following tasks:
\begin{enumerate}
\item Formalize the rules of logical inference rigorously, turning
  them into a comprehensible system of rules.
\item Prove, for a given axiomatic system $\meta{A}$ (whose
  consistency we're trying to determine), that no contradiction can be
  deduced from $\meta{A}$.
\end{enumerate}

\paragraph{Method of Arithmetization}
We don't have to repeat the same work for each axiomatic system.
Instead we can invoke the method of arithmetization (as alluded to
earlier). This can be characterized as follows:

We seek an axiomatic system $\meta{A}$ with the following property:
\begin{enumerate}
\item $\meta{A}$ is comprehensible to such a degree that we can
  determine no contradictions may be deduced from $\meta{A}$
\item $\meta{A}$ has to be sufficiently comprehensive, in the
  following sense: for any axiomatic system $\meta{B}$, we assume the
  existence of a model $\meta{C}$ of $\meta{A}$, and we can construct
  a model of $\meta{B}$ using the individuals from the model $\meta{C}$.
\end{enumerate}

\paragraph{}
This then proves the consistency of axiom system $\meta{B}$.

\subparagraph{Remark: G\"{o}del's Incompleteness}
As I interpret G\"{o}del's incompleteness theorems, the problem with
Hilbert and Bernays's plans thus outlined is the first step (stipulating
the consistency of $\meta{A}$). We thus are reduced to relative
consistencies ($\meta{B}$ is consistent provided $\meta{A}$ is consistent).

\paragraph{}
Hilbert and Bernays propose that arithmetic constitutes such a system
$\meta{A}$. 

\paragraph{}
This ``reduction'' of axiomatic theories to arithmetic does not depend
on arithmetic ``consists of intuitively producible facts''
(Mueller: ``being a set of facts presentable to intuition'').

Arithmetic doesn't need to be anything more than a ``formation of ideas''
(Mueller: ``ideal structure'') which
\begin{enumerate}
\item we can prove consistent, and
\item provides a systematic framework encompassing the axiom systems
  of theoretical sciences.
\end{enumerate}
We can then prove the consistency of these axiomatic systems of
theoretical sciences.

Mueller translation:
\begin{quote}
The ``reduction'' of axiomatic theories to arithmetic does not depend upon
arithmetic being a set of facts presentable to the intuition;
%
arithmetic need be no more than an ideal structure which we can prove
consistent and which provides a systematic framework encompassing the
axiom systems of the theoretical sciences;
%
because they are encompassed in this framework the idealizations of
what is actually given which they involve will also be proved
consistent.
\end{quote}

\paragraph{Summary of discussion}
Hilbert and Bernays now recaps the discussion thus far:
\begin{enumerate}
\item we can use the method of exhibition to prove the consistency of
  certain axiomatic systems;
\item when an axiomatic system cannot be satisfied by any finite
  domain of individuals, we run into problems using the method of exhibition.
\item This is because we cannot take for granted the existence of an
  infinite domain of individuals.
\item We must provide a proof of consistency for using an infinite
  domain of individuals.
\end{enumerate}

\paragraph{}
In light of the failure of the ``positive'' method of deciding
consistency, there is only one possibility left: we need to give a
proof of consistency in the negative sense, i.e., a
\emph{proof of impossibility}.
And this requires formalizing logical inference.

\paragraph{}
If we are going to approach the task (of giving such a proof of
impossibility), we must be clear it cannot be carried out using
axiomatic-existential methods of inference.

Instead, we must use only those kinds of inference which are free from
idealizing assumptions of existence.

\paragraph{}
As a result of this deliberation, the following thought comes to mind:
If this proof of impossibility can be carried out without
axiomatic-existential assumptions, then shouldn't it also be possible
to ``found'' [``ground''?] arithmetic directly in the same way and
make the proof of impossibility completely superfluous?

Hilbert and Bernays considers this question in the next chapter.
