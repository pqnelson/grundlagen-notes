\chapter{Elementary Number Theory. --- Finitistic Inference and its Limits}

\section{Method of intuitive consideration, its application in elementary number theory}

\subsection{Notion of a numeral; ``smaller than'' relation; addition}

\paragraph{} %  1

\paragraph{} %  2

\paragraph{} %  3

\paragraph{} %  4

\paragraph{} %  5

\paragraph{} %  6

\paragraph{} %  7

\paragraph{} %  8
The things we obtain, starting with 1, by applying this progression,
we obtain guys like
\begin{equation*}
1,\quad11,\quad111,\quad1111,\dots
\end{equation*}
The process is just appending ``1'' to the previous entry in the sequence.

\subparagraph{Finitistic property}
Critically, Hilbert and Bernays note that we have used a ``concrete
\emph{construction} which terminates'', and that this ``construction''
[\textit{Aufbau\/}] can therefore be reversed in terms of a
step-by-step \emph{decomposition} [\textit{Abbau\/}].

If we wanted to generalize Hilbert's programme from ``contentual
number theory'' as the metatheory to, say, some sort of statically
typed total functional programming language, then these properties are
crucial (and are satisfied by algebraic data types).

\paragraph{} %  9

\paragraph{} % 10

\paragraph{Communication signs disjoint from numerals} % 11
There is another class of signs which Hilbert and Bernays call
``communication signs''. These are fundamentally different than
numerals, because numerals constitute the objects of number theory.

\subparagraph{In $\FS0/$}
If I were to reflect this choice in $\FS0/$, it would make sense to
introduce a primitive atom encoding $1$ and then work with
tuples of $1$ as numerals. All other atoms would be ``communication signs''.

\paragraph{} % 12

\paragraph{} % 13

\paragraph{} % 14

\paragraph{} % 15

\paragraph{} % 16

\paragraph{} % 17

\paragraph{} % 18

\paragraph{} % 19

\paragraph{} % 20

\subsection[Laws of calculation]{Laws of calculation; mathematical induction; multiplication; divisibility; prime number}

\paragraph{} % 21

\paragraph{} % 22

\paragraph{} % 23

\paragraph{} % 24

\paragraph{} % 25

\paragraph{} % 26

\paragraph{} % 27

\paragraph{} % 28

\paragraph{} % 29

\paragraph{} % 30

\paragraph{} % 31

\paragraph{} % 32

\paragraph{} % 33

\paragraph{} % 34

\paragraph{} % 35

\paragraph{} % 36

\paragraph{} % 37

\paragraph{} % 38

\paragraph{} % 39

\paragraph{} % 40

\paragraph{} % 41

\paragraph{} % 42

\paragraph{} % 43

\paragraph{} % 44

\paragraph{} % 45

\paragraph{} % 46

\paragraph{} % 47

\section{Further applications of intuitive considerations}
% 48

\paragraph{} % 48

\paragraph{} % 49

\paragraph{} % 50

\paragraph{} % 51

\paragraph{} % 52

\paragraph{} % 53

\paragraph{} % 54

\paragraph{} % 55

\paragraph{} % 56

\paragraph{} % 57

\paragraph{} % 58

\paragraph{} % 59

\paragraph{} % 60

\paragraph{} % 61

\paragraph{} % 62

\paragraph{} % 63

\paragraph{} % 64

\paragraph{} % 65

\paragraph{} % 66

\paragraph{} % 67

\paragraph{} % 68

\section{The finitistic standpoint; transgression of this standpoint already in number theory}

\subsection{Logical characterization of the finitistic standpoint}
\paragraph{} % 69

Mueller~\cite[p.44]{mueller2006grundlagen} translates this paragraph
as:
\begin{quote}
The consideration of the principles of number theory and algebra has
served to show us the application and use of direct contentual
inference carried out in thought experiments performed on intuitively
imagined objects and free from axiomatic assumptions. We will call
this kind of inference ``[finitistic]'' [\textit{finit\/}] inference
in order to have a short expression; likewise we shall call the
attitude underlying this kind of inference the ``finitistic'' attitude
or point of view. We will speak of finitistic concepts or assertions
in the same sense; in using the word ``finitistic'' we convey the idea
that the consideration, assertion or definition in question remains
within the limits of objects which it is in principle possible to
observe and of processes which it is in principle possible to
complete; that it is carried out in the framework of concrete thought.
\end{quote}

\paragraph{} % 70

\paragraph{} % 71

\paragraph{} % 72

\paragraph{} % 73

\paragraph{} % 74

\paragraph{} % 75

\paragraph{} % 76

\paragraph{} % 77

\paragraph{} % 78

\paragraph{} % 79

\paragraph{} % 80

\paragraph{} % 81

\paragraph{} % 82

\paragraph{} % 83

\paragraph{} % 84

\subsection{The ``tertium non datur'' for integers; the least-integer principle}

\paragraph{} % 85

\paragraph{} % 86

\paragraph{} % 87

\paragraph{} % 88

\paragraph{} % 89

\paragraph{} % 90

\paragraph{} % 91

\paragraph{} % 92

\section{Non-finitistic methods in analysis}
% 93 or 95

\paragraph{} % 93

\paragraph{} % 94

\paragraph{} % 95

\paragraph{} % 96

\paragraph{} % 97

\paragraph{} % 98

\paragraph{} % 99

\paragraph{} % 100

\paragraph{} % 101

\paragraph{} % 102

\paragraph{} % 103

\paragraph{} % 104

\paragraph{} % 105

\paragraph{} % 106

\paragraph{} % 107

\paragraph{} % 108

\paragraph{} % 109

\paragraph{} % 110

\paragraph{} % 111

\paragraph{} % 112

\paragraph{} % 113

\paragraph{} % 114

\paragraph{} % 115

\paragraph{} % 116

\paragraph{} % 117

\paragraph{} % 118

\section{Investigations on the direct finitistic grounding of arithmetic; return to the previous way of posing the problem; proof theory}
% 119

\paragraph{} % 119

\paragraph{} % 0

\paragraph{} % 1

\paragraph{} % 2

\paragraph{} % 3

\paragraph{} % 4

\paragraph{} % 5

\paragraph{} % 6

\paragraph{} % 7

\paragraph{} % 8

\paragraph{} % 9
