\chapter[Elementary Number Theory]{Elementary Number Theory. --- Finitistic Inference and its Limits}

\section{Method of intuitive consideration, its application in elementary number theory}

\subsection{Notion of a numeral; ``smaller than'' relation; addition}

\paragraph{} %  1
The question has been raised whether we can just ``directly'' address
the matter of existence and consistency.

It's worth mentioning that this existential axiomatic approach is not
``the original approach'' to mathematics.

\paragraph{Euclid} %  2
Euclid worked with a more contentual axiomatic method. The intuitive
meaning of ``figures'' [the basic notions] was clear in Euclid's
\emph{Elements}.

Further, its axioms are not in ``existential form'' either. There is
no presupposition that the domain of individuals consists of points
and lines.

Euclid does not work with existential axioms, but gives us
constructional postulates instead.

\paragraph{Constructional postulates} %  3
Such ``constructional postulates'' tell us \emph{it is always possible
to construct a line passing through two distinct points}, and \emph{it
is possible to construct a circle centered at a given point with a
given radius}.

\paragraph{} %  4
This only works if the postulates are viewed [\textit{angesehen werden\/}] as expressions [\textit{Austrucks\/}]
of a known actuality [\textit{Tats\"{a}chlichen\/}] or of direct
evidence [\textit{unmittelbaren Evidenz\/}].

``It is well known'' the question regarding ``the scope of validity of
geometry'' is contraversial (by which I think Hilbert and Bernays
means: there is no single model of geometry; instead there are many
\emph{non-Euclidean} geometries discovered in the 19th century).

Hilbert and Bernays argues that a substantial advantage of formal
axiomatics is that it makes the grounding of geometry
\emph{independent} of this question.

\paragraph{} %  5
We're not concerned with these types of issues in elementary number
theory. We don't need to worry about the character of knowledge of
geometric matters.

Furthermore, in elementary number theory (and elementary algebra), we
find ``the purest'' manifestation of the ``standpoint of direct
contentual thought'' which has evolved ``without axiomatic assumptions''.

\paragraph{} %  6
This methodological standpoint is characterized by ``thought
experiments'' [\textit{Gedankenexperimenten\/}] with things that are
assumed to be ``concretely present'' [\textit{konkrete vorgliegend\/}]
such as numbers in number theory (or expressions of letters with
numerical coefficients in elementary algebra).

\paragraph{} %  7
Hilbert and Bernays analyze this situation more closely in this
chapter.

In number theory, we have ``an initial thing'' and ``a process of progression''.
We must fix both intuitively in some definite way. It is inessential
how we do this, but the choice must be maintained.

Hilbert and Bernays choose ``1'' as ``the initial thing'', and
appending as a suffix ``1''.

\paragraph{} %  8
The things we obtain, starting with 1, by applying this progression,
we obtain guys like
\begin{equation*}
1,\quad11,\quad111,\quad1111,\dots
\end{equation*}
The process is just appending ``1'' to the previous entry in the sequence.

\subparagraph{``Figures''}
Hilbert and Bernays refers to the result of this construction process
as ``figures'' [German: \textit{Figur\/}]. This is the first time this
word appears in this chapter. I am uncertain what to make of its
significance. 

\subparagraph{Finitistic property}
Critically, Hilbert and Bernays note that we have used a ``concrete
\emph{construction} which terminates'', and that this ``construction''
[\textit{Aufbau\/}] can therefore be reversed in terms of a
step-by-step \emph{decomposition} [\textit{Abbau\/}].

If we wanted to generalize Hilbert's programme from ``contentual
number theory'' as the metatheory to, say, some sort of statically
typed total functional programming language, then these properties are
crucial (and are satisfied by algebraic data types).

\paragraph{} %  9
Openly announcing we're deviating from common usage of language,
Hilbert and Bernays refers to these figures as ``numerals''.

\paragraph{``Unambiguously recognizable'': tolerance of orthographics} % 10
Hilbert and Bernays announce a willing tolerance in the orthographics
of these figures. Presumably a certain degree of ambiguity arises when
writing with pen on paper (the spacing between the figures, the height
of the figures, the various possible serifs attached [or not] to
figures, etc.) which we should understand and allow.

They refer to this tolerance of orthographic variability a property
called ``unambiguously recognizable''.

\paragraph{Communication signs disjoint from numerals} % 11
There is another class of signs which Hilbert and Bernays call
``communication signs''. These are fundamentally different than
numerals, because numerals constitute the ``objects'' of number theory.

\subparagraph{In $\FS0/$}
If I were to reflect this choice in $\FS0/$, it would make sense to
introduce a primitive atom encoding $1$ and then work with
tuples of $1$ as numerals. All other atoms would be ``communication signs''.

\paragraph{} % 12
A ``symbol for communication'' by itself is also a figure.

Therefore symbols for communication should also be
``unambiguously recognizable''.

Within the theory (of elementary number theory), no ``symbol of communication''
is made into an ``object of consideration'', but it merely serves as a
means for concise, clear formulation of facts, assertions,
assumptions.

(This seems to be borrowing a similar notion from Russell and
Bernays's \textit{Principia Mathematica}.)

\paragraph{Symbols of communication in number theory} % 13
In elementary number theory, Hilbert and Bernays offers the following
symbols of communication:
\begin{enumerate}
\item ``small German letters'' [i.e., Fraktur letters] as variables
  for numerals
\item the customary number symbols as abbreviations for definite
  numerals (e.g., ``2'' for ``11'', ``3'' for ``111'', etc.)
\item symbols for certain formation processes and operations of
  calculation, which may be applied to definite as well as indefinite
  numerals (e.g., ``$\mathfrak{a}+11$'')
\item the symbol ``='' to communicate ``figural coincidence'', and the
  symbol ``$\neq$'' to communicate difference of numerals; the symbols
  ``$<$'' and ``$>$'' will be used for the relation of magnitude
  (which will be discussed in a minute)
\item parentheses as symbols for the order of processes, where this
  order is not clear without further explanation.
\end{enumerate}

\paragraph{} % 14
The way these symbols work and their contentual meanings will be made
clear as we work our way through elementary number theory.

\paragraph{Relation of magnitude} % 15
The first thing Hilbert and Bernays does: clarify the relation of
magnitude. Let numeral $\mathfrak{a}$ be different from
$\mathfrak{b}$. How is this possible?

Both must begin with ``1'' and iteratively append ``1'' finitely many
times. But one stops whereas the other continues; i.e., one is the
``initial segment'' of the other.

Since these are different numerals, one must be the \emph{proper}
initial segment of the other.

\paragraph{} % 16
If the numeral $\mathfrak{a}$ coincides with an initial segment of
$\mathfrak{b}$, then we say ``$\mathfrak{a}$ is smaller than $\mathfrak{b}$''
(or equivalently, ``$\mathfrak{b}$ is greater than $\mathfrak{a}$'').
For this, we use the notation
\begin{equation*}
\mathfrak{a}<\mathfrak{b},\qquad\mathfrak{b}>\mathfrak{a}.
\end{equation*}

\subparagraph{Strict and total}
It follows from the previous consideration that for any numerals
$\mathfrak{a}$ and $\mathfrak{b}$, exactly one of the following must
always hold:
\begin{enumerate}
\item $\mathfrak{a} < \mathfrak{b}$,
\item $\mathfrak{a} = \mathfrak{b}$,
\item $\mathfrak{a} > \mathfrak{b}$.
\end{enumerate}
Moreover, these are mutually exclusive possibilities.

\subparagraph{Transitivity}
Further, ``it is obvious'' that
if $\mathfrak{a}<\mathfrak{b}$ and $\mathfrak{b}<\mathfrak{c}$,
then $\mathfrak{a}<\mathfrak{c}$.

I think this is not true, I think that we could prove this claim. But
I am uninterested in the tedium needed to prove it.

\paragraph{Addition operator} % 17
When we have numerals $\mathfrak{b}<\mathfrak{a}$, we can identify the
initial segment of $\mathfrak{a}$ which coincides with
$\mathfrak{b}$. Then we can identify a numeral $\mathfrak{c}$ which
coincides with the remaining segment of $\mathfrak{a}$.

If we're given $\mathfrak{b}$ and $\mathfrak{c}$, then we can
construct a new numeral by appending $\mathfrak{c}$ after
$\mathfrak{b}$. This defines addition, and we write
$\mathfrak{b}+\mathfrak{c}$ for this operation.

\paragraph{} % 18
If a numeral $\mathfrak{b}<\mathfrak{a}$ coincides with a segment of
$\mathfrak{a}$, then the remaining segment of $\mathfrak{a}$ is again
a numeral $\mathfrak{c}$. 

In this case, we have a representation for $\mathfrak{a}$
of the form $\mathfrak{b}+\mathfrak{c}$, where $\mathfrak{c}$ is a
numeral.

Moreover, we have
\begin{equation*}
\mathfrak{b} < \mathfrak{b} + \mathfrak{c}.
\end{equation*}

For a concrete example, ``2'' mean ``11'' and ``3'' means ``111'', and
``5'' means ``11111''. We see that ``$2 + 3$'' stands for appending
``111'' (i.e., three appending operations of ``1'') to ``11'', giving
us ``11111''. And that's precisely ``5''.

\paragraph{Example of a false proposition} % 19
The proposition ``$2+3=4$'' is an example of a false proposition.

\subsection[Laws of calculation]{Laws of calculation; mathematical induction; multiplication; divisibility; prime number}

\paragraph{} % 20
Hilbert and Bernays now turn their attention to proving the validity
of the laws of calculation involving this newly defined ``addition''
operation.

\paragraph{} % 21
The ``laws of calculation'' are regarded as sentences involving
arbitrary numerals. They are seen to hold by ``intuitive consideration''.

I am taking this to mean that the ``discussions'' which follow are
examples of what qualifies as an ``intuitive consideration''.

\paragraph{} % 22

\paragraph{} % 23

\paragraph{} % 24

\paragraph{} % 25

\paragraph{} % 26

\paragraph{} % 27

\paragraph{} % 28

\paragraph{} % 29

\paragraph{} % 30

\paragraph{} % 31

\paragraph{} % 32

\paragraph{} % 33

\paragraph{} % 34

\paragraph{} % 35

\paragraph{} % 36

\paragraph{} % 37

\paragraph{} % 38

\paragraph{} % 39

\paragraph{} % 40

\paragraph{} % 41

\paragraph{} % 42

\paragraph{} % 43

\paragraph{} % 44

\paragraph{} % 45

\paragraph{} % 46

\paragraph{} % 47

\section{Further applications of intuitive considerations}
% 48

\paragraph{} % 48

\paragraph{} % 49

\paragraph{} % 50

\paragraph{} % 51

\paragraph{} % 52

\paragraph{} % 53

\paragraph{} % 54

\paragraph{} % 55

\paragraph{} % 56

\paragraph{} % 57

\paragraph{} % 58

\paragraph{} % 59

\paragraph{} % 60

\paragraph{} % 61

\paragraph{} % 62

\paragraph{} % 63

\paragraph{} % 64

\paragraph{} % 65

\paragraph{} % 66

\paragraph{} % 67

\paragraph{} % 68

\section[The finitistic standpoint]{The finitistic standpoint; transgression of this standpoint already in number theory}

\subsection{Logical characterization of the finitistic standpoint}
\paragraph{} % 69

The previous sections in this chapter was meant to show-by-example how
to apply and implement ``direct contentual inference'' [that takes
  place in] thought-experiments on ``intuitively conceived objects''
and is ``free of axiomatic assumptions''.

Hilbert and Bernays call this kind of inference ``finitistic''.

(This paragraph is a couple gigantic sentences in the German original.)

\subparagraph{Finitistic attitude}
Hilbert and Bernays call the methodological attitude underlying this
kind of inference as the ``finitistic'' attitude or the ``finitistic''
standpoint.

In the same sense, they speak of ``finitistic concept formations'' and
``finitistic assertions''.

\subparagraph{Finitistic criteria}
Each use of the word ``finitistic'' conveys the idea that the relevant
consideration, assertion, or definition is confined to
\begin{enumerate}
\item objects that are conceivable in principle\footnote{The German
\textit{vorstellbarkeit} translates to ``conceivable'' or
``imaginable'', fairly unambiguously: \textit{vorstell-} ``imagine,
conceive'' + \textit{-bar} ``-able''. Here
\textit{grunds\"{a}tzlichen} means ``based on principle'', but can
also mean ``fundamental, elementary, principal''.}
  [Ger.: \textit{der grunds\"{a}tzlichen vorstellbarkeit von objekten\/}], and
\item processes that can be ``effectively executed'' [effectively computable]\footnote{The
German \textit{ausf\"{u}hrbarkeit} is too weird for me to grasp. It
derives from \textit{ausf\"{u}hrbar} ``executable, feasible'' but also
``exportable'', which itself derives from \textit{ausf\"{u}hren} [to
  carry out, execute, elaborate on, explain] + \textit{-bar}
[-able]. The \textit{-keit} [-ness] suffix appears to have
significance beyond me, it's a variant form of
\textit{-heit}. Wikipedia's German page for ``Algorithm'' has the
second requirement be ``\textit{Jeder Schritt des Verfahrens muss
  tats\"{a}chlich ausf\"{u}hrbar sein (Ausf\"{u}hrbarkeit).\/}''
(English: Each step of the procedure must actually be executable (feasibility).).} [Ger.: \textit{ausf\"{u}hrbarkeit}] in principle [Ger: \textit{grunds\"{a}tzlichen}],\footnote{The second criterion in original Ger.: \textit{der grunds\"{a}tzlichen ausf\"{u}hrbarkeit von Prozessen}}
\end{enumerate}
and thus it remains within the scope of a concrete treatment [Ger.:
  \textit{und sich somit im Rahmen konkreter Betrachtung
    vollzieht\/}].\footnote{The phrase ``\textit{und sich somit\/}''
means ``and thus''; \textit{somit\/} means ``remains'' or ``takes
place in''; \textit{im Rahmen} means ``as a part of'', ``in the scope of''; \textit{Betrachtung\/} means
``treatment'' or ``consideration'', \textit{konkreter Betrachtung\/}
means ``concrete treatment'' (or ``concrete consideration''); \textit{vollzieht\/} means ``take place''.}
The
German original for the itemization and final phrase:
\begin{quote}
\dots der grunds\"{a}tzlichen vorstellbarkeit von objekten sowie der von objekten sowie der grunds\"{a}tzlichen ausf\"{u}hrbarkeit von Prozessen h\"{a}lt und sich somit im Rahmen konkreter Betrachtung vollzieht.
\end{quote}
The itemization seems clear in spirit by the repeated use of
``\textit{der\/}''.

\subparagraph{Finitistic = Computable + Constraint}
Also note that the second criterion, I think, anachronistically reads as
``computable processes''. In which case, it's the first criterion
(``objects are conceivable in principle'') which restricts
``finitary'' as a proper subset of ``computable''.

But it is unclear to me which criterion distinguishes finitism from
constructivism. I would hazard to guess it's the first criterion,
since the second criterion is satisfied by intuitionistic higher-order
logic.

Also note that page 43, \pilcrow\ref{ch2:par124}, Brouwer's
intuitionism is discussed as an extension of finitism allowing for
inferences on propositions without knowing their intuitive
content. This seems to suggest Brouwer's intuitionism violates the
first criterion (``objects are conceivable in principle''), which
supports my hypothesis that finitism is demarcated as intuitionism
restricted by an additional criterion.

\subparagraph{Alternative translation}
Mueller~\cite[p.44]{mueller2006grundlagen} translates this paragraph
as:
\begin{quote}
The consideration of the principles of number theory and algebra has
served to show us the application and use of direct contentual
inference carried out in thought experiments performed on intuitively
imagined objects and free from axiomatic assumptions. We will call
this kind of inference ``[finitistic]'' [\textit{finit\/}] inference
in order to have a short expression; likewise we shall call the
attitude underlying this kind of inference the ``finitistic'' attitude
or point of view. We will speak of finitistic concepts or assertions
in the same sense; in using the word ``finitistic'' we convey the idea
that the consideration, assertion or definition in question remains
within the limits of objects which it is in principle possible to
observe and of processes which it is in principle possible to
complete; that it is carried out in the framework of concrete thought.
\end{quote}

\paragraph{} % 70
Hilbert and Bernays now will emphasize aspects of ``the usage of
logical forms of judgement in finitistic thinking'' by working on
propositions on numerals as examples.

\paragraph{Universal judgements} % 71
A ``universal judgement'' [Ger.: \textit{allgemeines
    Urteil\/}\footnote{Recall, since Kant, the German word
  \textit{Urteil\/} has been used as ``judgement'' in Logic.}] about numerals can be given a finitistic
interpretation ``only in a hypothetical sense'' [Ger.: \textit{hypothetisches sinn\/}],
as a proposition about every arbitrary [Ger.: \textit{jedwede\/}]
numeral that is effectively given/presented [Ger.: \textit{vorgelegte\/}].

A universal judgement expresses a law [Ger.: \textit{Gesetz\/}] which
has to be proven true for each individual case [Ger.: \textit{Einzelfall\/}].

\paragraph{Existential judgements} % 72
Hilbert and Bernays call existential judgements a kind of ``partial
judgement'' [Ger.: \textit{Partialurteil\/}]. They say it is an
``incomplete communication'' [Ger.: \textit{unvollst\"{a}ndige Mitteilung\/}]
of a ``more precisely determined'' statement [Ger.: \textit{genauer bestimmten aussage\/}].
It seems that they mean it is a sort of argot, a stage instruction.

For example, ``There exists a numeral $\mathfrak{n}$ such that
$\mathfrak{A}(\mathfrak{n})$'' consists either in the direct
presentation of a numeral satisfying the predicate, or presenting a
procedure for constructing such a numeral. In the latter case, the
procedure has to terminate after finitely many steps (Hilbert and
Bernays say that it has to include a definite limit to the number of
steps). --- I can't help but read this as ``the procedure must be a
total function''.

\subparagraph{Partial function?}
It's unclear to me how to interpret this discussion. Should we think
of an existential judgement as a partial function on numerals? Is that
what Hilbert and Bernays mean by ``incomplete communication of a procedure''?

\paragraph{Combined universal and existential judgements} % 73
Mixed judgements involving both universal and existential judgements
are interpreted as one might expect. For example, ``For any numeral
$\mathfrak{a}$ satisfying $\mathfrak{A}(\mathfrak{a})$, there exists a
numeral $\mathfrak{b}$ such that
$\mathfrak{B}(\mathfrak{a},\mathfrak{b})$ holds.''

We finitistically interpret this as an ``incomplete communication'' of
a procedure which, given any numeral $\mathfrak{a}$ satisfying the condition
$\mathfrak{A}(\mathfrak{a})$, permits us to find a numeral
$\mathfrak{b}$ such that $\mathfrak{B}(\mathfrak{a},\mathfrak{b})$ holds.

\paragraph{Negation} % 74
Negation requires great care, from the finitistic perspective, and so
several paragraphs will be dedicated to investigating it.

\paragraph{Negating ``Elementary'' Judgements} % 75
Negating an ``Elementary'' [Ger.: \textit{elementaren\/}] judgement
concerning a question decidable by direct intuitive observation, is
straightforward.

The example given: if $\mathfrak{k}$ and $\mathfrak{l}$ are definite
numerals, then it can be directly observed whether
$\mathfrak{k}+\mathfrak{k}=\mathfrak{l}$ is the case or not.

\paragraph{} % 76
The negation of elementary judgements merely requires checking if an
``intuitive decision'' deviates from the state of affairs described by
the judgement.

\subparagraph{Excluded middle for elementary judgements}
Curiously, Hilbert and Bernays writes, ``and for an elementary
judgement it may be taken for granted that either the judgement itself
or its negation holds.'' (Pg 33) In other words, the law of the
excluded middle holds for elementary finitistic judgements.

\paragraph{Negating quantified judgements} % 77
For a universal or existential judgement (i.e., a judgement with
quantifiers), it is not immediately obvious what should count as its
negation in the finitistic sense.

\paragraph{Negating existential judgements} % 78
For an existential assertion, what is its negation? Negating ``there is a
numeral $\mathfrak{n}$ with property $\mathfrak{A}(\mathfrak{n})$''
may be meant in a vague [\textit{unscharfem\/}] sense [\textit{Sinne\/}]: as the statement that we do not have a
numeral with this property at our disposal (i.e., that a number with this property is not available to us for specification).
Such a statement, however, has no ``objective significance''
[\textit{objecktive Bedeutung\/}] because
it is relative to an ``accidental state of knowledge''.

However, if we wish to assert the unavailablility of a numeral
$\mathfrak{n}$ with the property $\mathfrak{A}(\mathfrak{n})$
independently of our state of knowledge [\textit{Erkenntniszustand\/}], then one can express this
finitistically only by an assertion of impossibility (stating a
numeral $\mathfrak{n}$ \emph{cannot} have property $\mathfrak{A}(\mathfrak{n})$).

\paragraph{} % 79


\paragraph{} % 80
Unlike negating elementary judgements, the existential proposition and
its [finitistic] negation are not propositions about the only two
possible results of \emph{one and the same decision}. Instead they
correspond to two distinct epistemic possibilities: the discovery of a
numeral with the given property, and the insight into the law about
numerals.

Claus-Peter Wirth helpfully notes that this appears to mean that the
negation of $\Exists{x}{\mathfrak{A}(x)}$ is
$\Not{\Exists{x}{\mathfrak{A}(x)}}=\Forall{x}{\Not{\mathfrak{A}(x)}}$.
What is different from the classical situation, however, is that in
general $\Forall{x}{\Not{\mathfrak{A}(x)}}\lor\Exists{x}{\mathfrak{A}(x)}$
does not hold in the finitistic sense.

\paragraph{Excluded law fails here} % 81
Hilbert and Bernays note that the law of the excluded middle fails to
holds with negating existential judgements.

\paragraph{Negating universal judgements} % 82
For a judgement of the form ``For every numeral $\mathfrak{n}$ we have
$\mathfrak{A}(\mathfrak{n})$'', it does not yet make finitistic sense
to negate such a thing. The validity of $\mathfrak{A}(\mathfrak{n})$
can be refuted with a counterexample, but this no longer constitutes
the contradictory opposite of the universal judgement.

\paragraph{} % 83
Finding a counterexample is not the only way to refute a universal
judgement. When deducing the consequences of a universal judgement, we
might run into a contradiction in some other way. This does not
eliminate the difficulty, it just complicates things.

Namely it is neither logically apparent that a universal judgement on
numerals must either hold or else lead to a contradiction in its
consequences (i.e., is refutable)\footnote{For example, an incomplete
theory will have universal judgements which are neither provable nor refutable.}, nor is it self-evident that such a
judgement --- if it is refutable --- is refutable with a
counterexample.\footnote{Claus-Peter Wirth notes this is true with
classical axiomatic theories as well.}

\paragraph{} % 84
The reason for negation being this complicated is all Brouwer's fault,
blame him. Brouwer argues that it's invalid to apply the law of
excluded middle for ``infinite totalities''. Finitistic negation
inherits this quality.

\subsection{The ``tertium non datur'' for integers; the least-integer principle}

\paragraph{} % 85

\paragraph{} % 86

\paragraph{} % 87

\paragraph{} % 88

\paragraph{} % 89

\paragraph{} % 90

\paragraph{} % 91

\paragraph{} % 92

\section{Non-finitistic methods in analysis}
% 93 or 95

\paragraph{} % 93

\paragraph{} % 94

\paragraph{} % 95

\paragraph{} % 96

\paragraph{} % 97

\paragraph{} % 98

\paragraph{} % 99

\paragraph{} % 100

\paragraph{} % 101

\paragraph{} % 102

\paragraph{} % 103

\paragraph{} % 104

\paragraph{} % 105

\paragraph{} % 106

\paragraph{} % 107

\paragraph{} % 108

\paragraph{} % 109

\paragraph{} % 110

\paragraph{} % 111

\paragraph{} % 112

\paragraph{} % 113

\paragraph{} % 114

\paragraph{} % 115

\paragraph{} % 116

\paragraph{} % 117

\paragraph{} % 118

\section[Direct Finitistc Grounding of Arithmetic]{Investigations on the direct finitistic grounding of arithmetic; return to the previous way of posing the problem; proof theory}
% 119

\paragraph{} % 119

\paragraph{} % 120

\paragraph{} % 121

\paragraph{} % 122

\paragraph{} % 123

\paragraph{}\label{ch2:par124}\ignorespaces% 124

\paragraph{} % 5

\paragraph{} % 6

\paragraph{} % 7

\paragraph{} % 8

\paragraph{} % 9
